\section{Patterns}
\subsection{AdapterView}
\textbf{1. Definieren/Einbinden der ListViews in einer Activity-XML Datei}
\begin{lstlisting}[language=xml]
<ListView
  android:id="@+id/listView"
  android:layout_width="match_parent"
  android:layout_height="match_parent"/>
\end{lstlisting}
\textbf{2. Rowlayout definieren in res/layout/rowlayout.xml}
\begin{lstlisting}[language=xml]
<!-- Hier koennte ein Icon, CheckBox, etc. sein -->
<TextView
  android:layout_width="match_parent"
  android:layout_height="match_parent"
  android:id="@+id/list_text"
  android:textAlignment="center"
  android:text="Platzhalter" />
\end{lstlisting}
\textbf{3. Instanziieren des Adapters}
\begin{lstlisting}[language=java]
ArrayAdapter<String> myArrayAdapter =
  new ArrayAdapter<>(
   this, // Die aktuelle Umgebung (diese Activity)
   R.layout.rowlayout, // Die ID des Zeilenlayouts (der XML-Layout Datei)
   R.id.list_text,   // Die ID eines TextView-Elements im Zeilenlayout
   names);  // Beispieldaten in einer ArrayList
\end{lstlisting}
\textbf{4. Verbinden des Adapters mit dem ListView}
\begin{lstlisting}[language=java]
ArrayList<String> names = new ArrayList<String>();
names.add("Severin"); names.add("Julian"); names.add("Sharon");
ListView myListView = (ListView) findViewById(R.id.listView);
myListView.setAdapter(myArrayAdapter);
\end{lstlisting}

\subsection{Recycler View} 
\begin{itemize}
\item \textbf{Adapter:} Analog zu eigenem ArrayAdapter
\item \textbf{ViewHolder:} Klasse welche die Views zwischenspeichert
\item \textbf{LayoutManager:} Defaults für übliche Layouts (optional)
\item \textbf{ItemDecorator:} Trennlinien oder andere Dekoratoren um die Elemente (optional)
\item \textbf{ItemAnimator:} Animiert hinzufügen, entfernen und scrolling von Elementen (optional)
\end{itemize}

Die RecyclerView bemerkt Änderungen an den Daten nicht von selbst. Um eine Aktualisierung zu veranlassen:
\begin{itemize}
  \item \code{notifyDataSetChanged()} //Holzhammer-Methode --> Overkill wenn nur zwei Elemente geändert
  \item \code{notifyItemChanged(positionToUpdate)}
  \item \code{notifyItemInserted(insertPosition)}
  \item \code{notifyItemRemoved(removePosition)}
\end{itemize}
Diese gibts auch noch in einer \code{notifyItemRange}-Variante

% \paragraph{Swipe Refresh} Anstelle, dass unsere App sich selbst aktualisiert kann man das mit einem Swipe Refresh machen. Dafür braucht man eine Support Library.
% \begin{lstlisting}[language=xml]
% <android.support.v4.widget.SwipeRefreshLayout>
%   <android.support.v7.widget.RecyclerView/>
% </android.support.v4.widget.SwipeRefreshLayout>
% \end{lstlisting}
% \begin{lstlisting}[language=java]
% final SwipeRefreshLayout swipe = 
%   (SwipeRefreshLayout) findViewById(R.id.swipeRefreshLayout);
% swipe.setOnRefreshListener(new SwipeRefreshLayout.OnRefreshListener() {
%   public void onRefresh() {
%   /*...*/
%   swipe.setRefreshing(false);
%   }
% });
% \end{lstlisting}
