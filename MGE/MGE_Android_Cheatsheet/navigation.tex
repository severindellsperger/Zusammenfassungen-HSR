\section{Fragment}
\begin{itemize}
    \item Möglicher Eintrittspunkt? Ja? --> zwingend Activity
    \item Möglicher Eintrittspunkt? Nein? --> Activity or Fragment
\end{itemize}
Ein Fragment ist ein modularer Teil einer Activity mit eigenem Lebenszyklus.Ein Fragment kann in mehreren Activities eingebunden werden und eine Activity kann mehrere Fragments beinhalten. //
\textbf{Fragment definieren}
\begin{lstlisting}[language=java]
public class MainActivityFragment extends Fragment {
  public MainActivityFragment() {}
  @Override
  public View onCreateView(LayoutInflater inflater, ViewGroup container, Bundle savedInstanceState) {
  return inflater.inflate(R.layout.fragment_main, container, false);
  // false bedeutet, dass die View nicht in den Container eingefuegt werden soll, da das bereits das Fragment macht
  }
  @Override
    public void onViewCreated(View view, Bundle savedInstanceState) {
        // Setup any handles to view objects here
        // EditText etFoo = (EditText) view.findViewById(R.id.etFoo);
    }   
}
\end{lstlisting}
\textbf{statisches Einfügen}
\begin{lstlisting}[language=xml]
<LinearLayout ...>
 <fragment
  android:id="@+id/fragment"
  android:name="com.example.myfragmentapplication.MainFragment"
  android:layout_width="match_parent"
  android:layout_height="match_parent"
  tools:layout="@layout/fragment_main"/>
</LinearLayout> 
\end{lstlisting}
\begin{lstlisting}[language=java]
public class MainActivity extends AppCompatActivity {
 @Override
 protected void onCreate (Bundle savedInstanceState) {
  super.oncreate(savedInstanceState);
  setContentView(R.layout.activity_main);
 }
}
\end{lstlisting}

\textbf{dynamisches Einfügen}
\begin{lstlisting}[language=xml]
<Linear Layout ...>
 <FrameLayout
   android:id="@+id/fragment_container"
   android:layout_width="match_parent"
   android:layout_height="match_parent" />
\end{lstlisting}
\begin{lstlisting}[language=java]
public class MainActivity extends Activity {
 @Override
 protected void onCreate(Bundle savedInstanceState) {
  super.onCreate(savedInstanceState);
  setContentView(R.layout.activity_main);
  FragmentTransaction ft = getSupportFragmentManager().beginTransaction();
  ft.add(R.id.fragment_container, new MainFragment());
  // or ft.replace(R.id.fragment_container, new MainFragment()); 
  ft.commit();
 }
}
\end{lstlisting}

\textbf{Activity-Fragment Kommunikation}
Best Practice ist, dass Fragments ein Interface zur Kommunikation definieren, welches Parent-Activity implementieren muss. So bleibt die Wiederverwendbarkeit des Fragments erhalten. Ebenfalls hat das Fragment so keine Abhängigkeit auf die Activity (d.h es gibt kein Problem beim Cast).
\begin{lstlisting}[language=java]
public interface OnItemSelected {
    void onItemSelected(String item);
}
public class MainActivityFragment extends Fragement {
  OnItemSelected callback;
  @Override
  public void onAttach(Context activity) {
    super.onAttach(activity);
    if (!(activity instanceof OnItemSelected)
        throw new AssertionError( 
            "Activity must implement OnItemSelectedlistener!");
    }
    callback = (OnItemSelected) activity;
  }
}
\end{lstlisting}

\includegraphics[scale=0.5]{fragment_lifecycle.png}
\begin{itemize}
    \item onAttach: Fragment wird einer Activity hinzugefügt
    \item onCreateView: UI des Fragments erstellen
    \item onActivityCreated: Wenn Activity onCreate fertig ist
    \item onDestroyView: Gegenstück zu onCreateView
    \item onDetach: Gegenstück zu onAttach
\end{itemize}

\textbf{Fragment mit ID finden}
\begin{lstlisting}
public class MainActivity extends AppCompatActivity {
    @Override
    public void onCreate(Bundle savedInstanceState) {
        super.onCreate(savedInstanceState);
        if (savedInstanceState == null) {
          DemoFragment fragmentDemo = (DemoFragment) 
              getSupportFragmentManager().findFragmentById(R.id.fragmentDemo);
        }    }    }
\end{lstlisting}















\paragraph{Master-Detail Navigation} Die Unterscheidung zwischen den Anzeigevarianten wird über Layouts getroffen. Das Default-Layout der \code{ItemListActivity} beinhaltet nur das \code{ItemListFragment}. Das Tablet enthält ein Linear-Layout mit dem \code{ItemListFragment} sowie ein Platzhalter für das \code{ItemDetailFragment}. Das Kriterium für das Tablet-Layout ist \textit{sw600dp}(smallest width 600dp).
\begin{lstlisting}[language=java]
public clas ItemListActivity extends Activity implements ItemListFragment.Callbacks {
  private boolean twoPane;
  @Override
  protected void onCreate(Bundle savedInstanceState) {
  super.onCreate(savedInstanceState);
  setContentView(R.layout.activity_item_list);
  if(findViewById(R.id.item_details_container) != null) {
    twoPane = true;
  }
  }
  @Override
  public void onItemSelected(String id) {
  if(twoPane) {
    Bundle arguments = new Bundle();
    arguments.putString(ItemDetailFragment.ARG_ITEM_ID, id);
    ItemDetailFragment fragment = new ItemDetailFragment();
    fragment.setArguments(arguments);
    getSupportFragmentManager()
    .beginTransaction()
    .replace(R.di.item_detail_container, fragment)
    .commit();
    //Fragment wird ausgetauscht
  } else {
    Intent detailIntent = new Intent(this, ItemDetailActivity.class);
    detailIntent.putExtra(ItemDetailFragment.ARG_ITEM_ID, id);
    startActivity(detailIntent);
    // Neue Activity wird gestartet
  }
  }
}
\end{lstlisting}
Activities sollte man verwenden, wenn man einen Einstiegspunkt in einen Task braucht. \textbf{Wichtig:} Activities leben weiter, wenn wir eine neue Activity starten.

Neben \code{fragmentTransation.add} gibt es noch \code{fragmentTransaction.replace}. Es kann auch direkt \code{replace} aufgerufen werden, auch wenn noch keines ge-addet wurde. Beispiel siehe Spick ''Settings Page''.

\paragraph{Callback}
Damit Fragments mit der Activity kommunizieren können:
\begin{enumerate}
  \item Interface definieren. Am besten ''event-mässig'', also \code{onEVENT}
  \item Interface auf Activity implementieren
  \item in Fragment im \code{onAttach}:\\ \code{if !(activity instanceof INTERFACE) throw new AssertionError}
\end{enumerate}

\subsection{Verschachtelung}
Ab Android 4.2, API-Level 17, können Fragments verschachtelt werden. Dann muss man mit dem \code{getChildFragmentManager} gearbeitet werden, z.B. beim ViewPager.

\subsection{Menus}
\includegraphics[scale=0.25]{Menus.png}
\paragraph{Options Menu} Das Options Menu ist teil der ActionBar. Es enthält Actions die generell für die App/Activity gedacht sind. Apps mit Navigation Drawer haben teilweise kein Options Menu mehr. \\
\textbf{Direkt im Java Code definieren}
\begin{lstlisting}[language=java]
public boolean onCreateOptionsMenu(Menu menu) {
  menu.add(0, START_MENU_ITEM, 0, "Start");
  menu.add(0, SUBMIT_MENU_ITEM, 0, "Submit");
  return true;
}
public boolean onOptionsItemSelected(MenuItem item){
  switch(item.getItemId()){
  case START_MENU_ITEM:
    // start
    return true;
  case SUBMIT_MENU_ITEM:
    // submit
    return true;
  }
  return super.onOptionsItemSelected(item);
}
\end{lstlisting}
\textbf{Besserer Ansatz (deklarativ) im XML
}\begin{lstlisting}[language=xml]
<menu xmlns:... tools:context=".MainActivity">
  <item android:id="@+id/action_search"
    android:title="@string/action_search"
    android:icon="@drawable/ic_action_search"
    android:orderInCategory="100"
    android:showAsAction="never" />
  <item android:id="@+id/action_settings"
    android:title="@string/action_settings"
    android:orderInCategory="100"
    android:showAsAction="never" />
</menu>
\end{lstlisting}
Menü muss noch eingebunden werden
\begin{lstlisting}[language=java]
public class MainActivity extends Activity {
  public boolean onCreateOptionsMenu(Menu menu) {
    getMenuInflater().inflate(R.menu.menu_main, menu);
    return true;
  }
  public boolean onOptionsItemSelected(MenuItem item) {
    switch (item.getItemId()) {
        case R.id.action_search: // ID aus dem main.xml
        // handle start
        ...
    }  }  }
\end{lstlisting}

Auch Fragments können Einträge dem Menu der Activity hinzufügen
\begin{lstlisting}[language=java]
public class MainActivityFragment extends Fragment {
    @Override
    public void onCreateOptionsMenu(Menu menu, MenuInflater inflater) {
    inflater.inflate(R.menu.menu_main, menu); }
    @Override
    public void onCreate(Bundle savedInstanceState) {
    super.onCreate(savedInstanceState);
    setHasOptionsMenu(true); }
\end{lstlisting}

% \paragraph{Settings-Page} Die Settings-Page erbt von der Klasse \code{PreferenceActivity}. Sie kann im \code{PreferenceScreen} Tag definiert werden.
% \begin{lstlisting}[language=xml]
% <!-- preferences.xml -->
% <PreferenceScreen xmlns:android="...">
%   <CheckBoxPreference
%   android:defaultValue="true"
%   android:key="@string/sync_pref"
%   android:title="Synchronize" />
%   <MultiSelectListPreference
%   android:entries="@array/languages"
%   android:entryValues="@array/languages"
%   android:key="@string/lang_pref"
%   android:title="Languages" />
% </PreferenceScreen>
% <!-- resources.xml -->
% <resources>
%   <string name="lang_pref">language</string>
%   <string name="sync_pref">synchronize</string>
%   <string-array name="languages">
%   <item>German (CH)</item>
%   <item>English (UK)</item>
%   </string-array>
% </resources>
% \end{lstlisting}
% \begin{lstlisting}[language=java]
% public class Preferences extends PreferenceActivity {
%   @Override
%   public void onCreate(Bundle savedInstanceState) {
%   super.onCreate(savedInstanceState);
%   getSupportFragmentManager()
%     .beginTransaction()
%     .replace(R.id.content, new PrefsFragment()
%     .commit();
%   PreferenceManager.setDefaultValues(this, R.xml.preferences, false);
%   }
%   public static class PrefsFragment extends PreferenceFragment {
%   @Override
%   public void onCreate(Bundle savedinstanceState) {
%     super.onCreate(savedInstanceState);
%     addPreferencesFromResource(R.xml.preferences);
%   }
%   }
% }
% \end{lstlisting}
% In der MainActivity mit dem Settings-Menueintrag brauchen wir bloss noch die neue Activity zu starten.
% \begin{lstlisting}[language=java]
% @Override
% public boolean onOptionsItemSelected(MenuItem item){
%   int id = item.getItemId();
%   if(id == R.id.action_settings) {
%   startActivity(new Intent(this, Prefernces.class));
%   return true;
%   }
%   return super.onOptionSItemSelected(item);
% }
% \end{lstlisting}
% Die Einstellung wird mittels dieser Funktion ausgelesen.
% \begin{lstlisting}[language=java]
% SharedPreferences settings = getSharedPreferences(FILENAME, 
%   MODE_PRIVATE); // alternativ: MODE_MULTI_PROCESS
% SharedPreferences.Editor editor = settings.edit();
% editor.putBoolean("disabled", false)

% boolean isDisabled = settings.getBoolean("disabled", false)
% // false ist Defaultwert falls nicht existent

% editor.commit();
% \end{lstlisting}
% Auch Fragments können Einträge dem Menu der Activity hinzufügen. Die Behandlung erfolgt analog entweder im Fragment oder in der Activity.
% \begin{lstlisting}[language=java]
% public class MainFragment extends Fragment {
%   @Override
%   public void onCreateOptionsMenu(Menu menu, MenuInflater inflater) {
%   inflater.inflate(R.menu.menu_main, menu);
%   }
%   @Override
%   public void onCreate(Bundle savedInstanceState) {
%   super.onCreate(savedInstanceState);
%   setHasOptionsMenu(true);
%   }
% }
% \end{lstlisting}
\paragraph{Context und Popup Menu} Ein Context Menü für Actions welche die selektierte View betreffen. Meist ändert sich bei einer Selektion die ActionBar.\\
Popup Menüs lassen sich z.B an einen beliebigen Button binden (Vergleichbar mit einem Spinner/Combobox). \\
Popup Menus eignen sich für Overflow-Style Menu für Actions die Context spezifisch sind (beispielsweise Weiterleiten und Antworten bei Mail). 
\begin{lstlisting}[language=xml]
<ImageButton
  android:layout_width="wrap_content" 
  android:layout_height="wrap_content" 
  android:src="@drawable/ic_overflow_holo_dark"
  android:contentDescription="@string/descr_overflow_button"
  android:onClick="showPopup" />
\end{lstlisting}
\begin{lstlisting}[language=java]
public void showPopup(View v) {
  PopupMenu popup = new PopupMenu(this, v);
  MenuInflater inflater = popup.getMenuInflater();
  inflater.inflate(R.menu.actions, popup.getMenu());
  popup.show();
}
\end{lstlisting}
\paragraph{Action Bar} Die ActionBar war der erste Versuch eine normierte "{}Schnittstelle"{} für Aktionen in einer App zu kreieren. \textbf{deprecated!}

\paragraph{Toolbar} Die Toolbar soll ab Android 5.0 die ActionBar ablösen. Sie ist flexibler aber noch nicht so weit verbreitet. Um sie in älteren Android Versionen verwenden zu können ist eine Support-Library nötig.
\includegraphics[scale=0.35]{toolbar.png}
\begin{lstlisting}[language=xml]
<RelativeLayout xmlns:android="..." xmlns:app="..." xmlns:tools="..."
  android:layout_width="match_parent" android:layout_height="match_parent"
  tools:context=".MainActivity">
  <android.support.v7.widget.Toolbar
  android:id="@+id/toolbar"
  android:layout_width="match_parent"
  android:layout_height="wrap_content">
  </android.support.v7.widget.Toolbar>
  <fragment
  ...
  android:layout_below="@+id/toolbar"
  tools:layout="@layout/fragment_main" />
</RelativeLayout>
\end{lstlisting}
Damit die Toolbar als ActionBar funktioniert braucht es zusätzliche Konfiguration im Java Code
\begin{lstlisting}[language=java]
public class MainActivity extends AppCompatActivity {
  @Override
  protected void onCreate(Bundle savedInstanceState) {
  super.onCreate(savedInstanceState);
  setContentView(R.layout.activity_main);
  Toolbar toolbar = (Toolbar)findViewById(R.id.toolbar);
  setSupportActionBar(toolbar);
  }
}
\end{lstlisting}
\paragraph{Navigation Drawer} Dies ist eine Art Hauptmenü. Es wechselt zwischen Activities oder Fragments. Es können auch gewisse Einstellungen verändert werden (Toggle-Button). \\
\includegraphics[scale=0.35]{navigationdrawer.png} \\
Der Drawer benötigt eine Layoutdefinition und eine Menudeklaration. Das Layout der Activity ist rechts vom Navigation Drawer und braucht ein FrameLayout.
\includegraphics[scale=0.25]{NavDrawer.png}
1. Toolbar, 2. Content, 3. Header, 4. Menu-Items
\begin{lstlisting}[language=xml]
<android.support.v4.widget.DrawerLayout xmlns:android="..." xmlns:app="..." xmlns:tools="..."
  android:id="@+id/drawer_layout"
  android:fitsSystemWindows="true"
  tools:context=".MainActivity">
  <FrameLayout
  android:id="@+id/content"
  android:orientation="vertical">
  <android.support.v7.widget.Toolbar
    android:id="@+id/toolbar"/>
  <TextView ... />
  </FrameLayout>
  <android.support.design.widget.NavigationView
  android:id="@+id/navigation_view"
  android:layout_gravity="start"
  app:headerLayout="@layout/drawer_header"
  app:menu="@menu/drawer" />
</android.support.v4.widget.DrawerLayout>
\end{lstlisting}
In der Activity kann ein Listener registriert werden, welcher auf Menüpunkte des Drawers reagiert.
\begin{lstlisting}[language=java]
NavigationView view = (NavigationView)findViewById(R.id.navigation_view);
view.setNavigationItemSelectedListener(
  new NavigationView.OnNavigationItemSelectedListener() {
  @Override
  public boolean onNavigationItemSelected(MenuItem item) {
  item.setChecked(true);
  drawerLayout.closeDrawers();
  return true;
  }
});
\end{lstlisting}

\paragraph{Bottom Navigation View (aktueller Trend)}
\includegraphics[scale=0.4]{img/bottom_navigation_view.png}
\begin{lstlisting}
<android.support.design.widget.BottomNavigationView
  android:id="@+id/navigation"
  android:layout_width="match_parent"
  android:layout_height="wrap_content"
  android:layout_gravity="bottom"
  android:background="?android:attr/windowBackground"
  app:menu="@menu/navigation" /> 
\end{lstlisting}

\paragraph{Menüs}
Im Handler: \code{true} bedeutet ''behandelt''.

Von Fragments aus können auch Menü-Optionen hinzugefügt werden. Dazu gibt es auch dort \code{onCreateOptionsMenu(Menu menu, MenuInflater inflater)}, dort wird der Inflater also bereits mitgegeben.

In \code{onCreate} muss unbedingt \code{setHasOptionsMenusetHasOptionsMenu(true)} gesetzt werden, sonst wird wahrscheinlich das Callback nie aufgerufen
\paragraph{Toast und Snackbar} Toast sind kleine Feedbacknachrichten. 
\begin{lstlisting}[language=java]
// Toast
Toast toast = Toast.makeText(getActivity(), "Hello", Toast.LENGTH_SHORT);
toast.show();

// Snackbar
private void mkSnack() {
    Snackbar sb = Snackbar.make(content, "Hello", Snackbar.LENGTH_LONG);
    sb.setAction("Again", new View.OnClickListener() {
        @Override
        public void onClick(View v){
            mkSnack();
        }     });
    sb.show(); }
\end{lstlisting}